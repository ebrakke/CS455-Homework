\documentclass[11pt]{article}
\usepackage{amssymb,amsmath,amsthm,url,graphicx}
\usepackage{fancyhdr}

\def\shownotes{1}   % set 1 for version with author notes
                    % set 0 for no notes



%uncomment to get hyperlinks
%\usepackage{hyperref}

%%%%%%%%%%%%%%%%%%%%%%%%%%%%%%%%%%%%%%%%%%%%%%%%%%%%%%%%%%%%%%
%Some macros (you can ignore everything until "end of macros")

\topmargin 0pt \advance \topmargin by -\headheight \advance
\topmargin by -\headsep

\textheight 8.9in

\oddsidemargin 0pt \evensidemargin \oddsidemargin \marginparwidth
0.5in

\textwidth 6.5in

%%%%%%

\providecommand{\vs}{vs. }
\providecommand{\ie}{\emph{i.e.,} }
\providecommand{\eg}{\emph{e.g.,} }
\providecommand{\cf}{\emph{cf.,} }
\providecommand{\etc}{\emph{etc.} }

\newcommand{\getsr}{\gets_{\mbox{\tiny R}}}
\newcommand{\bits}{\{0,1\}}
\newcommand{\bit}{\{0,1\}}
\newcommand{\Ex}{\mathbb{E}}
\newcommand{\eqdef}{\stackrel{def}{=}}
\newcommand{\To}{\rightarrow}
\newcommand{\e}{\epsilon}
\newcommand{\R}{\mathbb{R}}
\newcommand{\N}{\mathbb{N}}
\newcommand{\Gen}{\mathsf{Gen}}
\newcommand{\Enc}{\mathsf{Enc}}
\newcommand{\Dec}{\mathsf{Dec}}
\newcommand{\Sign}{\mathsf{Sign}}
\newcommand{\Ver}{\mathsf{Ver}}

\providecommand{\mypara}[1]{\smallskip\noindent\emph{#1} }
\providecommand{\myparab}[1]{\smallskip\noindent\textbf{#1} }
\providecommand{\myparasc}[1]{\smallskip\noindent\textsc{#1} }
\providecommand{\para}{\smallskip\noindent}


\newtheorem{theorem}{Theorem}
\theoremstyle{definition}
\newtheorem{ex}{Exercise}
\newtheorem{definition}{Definition}

%%%%%%%  Author Notes %%%%%%%d
%
\ifnum\shownotes=1
\newcommand{\authnote}[2]{{ $\ll$\textsf{\footnotesize #1 notes: #2}$\gg$}}
\else
\newcommand{\authnote}[2]{}
\fi
\newcommand{\Snote}[1]{{\authnote{Solution}{#1}}}
\newcommand{\Inote}[1]{{\authnote{Solution}{#1}}}
\newcommand{\Ichanged}[1]{{\authnote{Changed}{#1}}}
%%%%%%%%%%%%%%%%%%%%%%%%%%%%%%%%%

\newcommand{\VAR}{\mathrm{VAR}}



% end of macros
%%%%%%%%%%%%%%%%%%%%%%%%%%%%%%%%%%%%%%%%%%%%%%%%%%%%%%%%%%%%%%


% page counting, header/footer
\usepackage{fancyhdr}
\usepackage{lastpage}
\pagestyle{fancy}
\lhead{\footnotesize \parbox{11cm}{CS455, Boston University, Fall 2015} }
\rhead{Erik Brakke}
\renewcommand{\headheight}{24pt}

\begin{document}

\title{Ethernet Lab}
\author{Erik Brakke}
\maketitle

\thispagestyle{fancy}
 
 
\section*{Answers}
\begin{enumerate}
	\item 10:08:b1:c1:6b:ad
	\item 10:c3:7b:43:62:39.  No, this is the address of the router
	\item 0x0800.  This corresponds to IP
	\item Byte number 67
	\item 10:c3:7b:43:62:39.  This is the address of the router
	\item 10:08:b1:c1:6b:ad.  Yes this is my computer
	\item 0x0800.  This is IP
	\item Byte number 73
	\item Address                  HWtype  HWaddress           Flags Mask            Iface\\
		  router.asus.com          ether   10:c3:7b:43:62:39   C                     wlan0\\
          Chromecast               ether   a4:77:33:02:2d:d8   C                     wlan0\\
 
	\item The source is 10:08:b1:c1:6b:ad and the destination is ff:ff:ff:ff:ff:ff
	\item The value is 0x0806 which is ARP
	\item It starts 20 bytes in.  The value is 1.  The ARP message does contain the IP address of the sender. This 'question', the MAC of the queried IP, appears 33 bytes into the message
	\item It appears 20 bytes in.  The value is 2 (reply).  The reply appears 33 bytes in. 192.168.1.1 is at 10:c3:7b:43:62:39
	\item The source is 10:c3:7b:43:62:39, the destination is 10:08:b1:c1:6b:ad
	\item It is possible that no one knows the answer to this request, or the wireshark capture was cut short before a response was given.


\end{enumerate}


\noindent\hrulefill


\section*{References}

None

\end{document} 