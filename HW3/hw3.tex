\documentclass[11pt]{article}
\usepackage{amssymb,amsmath,amsthm,url,graphicx}
\usepackage{fancyhdr}

\def\shownotes{1}   % set 1 for version with author notes
                    % set 0 for no notes



%uncomment to get hyperlinks
%\usepackage{hyperref}

%%%%%%%%%%%%%%%%%%%%%%%%%%%%%%%%%%%%%%%%%%%%%%%%%%%%%%%%%%%%%%
%Some macros (you can ignore everything until "end of macros")

\topmargin 0pt \advance \topmargin by -\headheight \advance
\topmargin by -\headsep

\textheight 8.9in

\oddsidemargin 0pt \evensidemargin \oddsidemargin \marginparwidth
0.5in

\textwidth 6.5in

%%%%%%

\providecommand{\vs}{vs. }
\providecommand{\ie}{\emph{i.e.,} }
\providecommand{\eg}{\emph{e.g.,} }
\providecommand{\cf}{\emph{cf.,} }
\providecommand{\etc}{\emph{etc.} }

\newcommand{\getsr}{\gets_{\mbox{\tiny R}}}
\newcommand{\bits}{\{0,1\}}
\newcommand{\bit}{\{0,1\}}
\newcommand{\Ex}{\mathbb{E}}
\newcommand{\eqdef}{\stackrel{def}{=}}
\newcommand{\To}{\rightarrow}
\newcommand{\e}{\epsilon}
\newcommand{\R}{\mathbb{R}}
\newcommand{\N}{\mathbb{N}}
\newcommand{\Gen}{\mathsf{Gen}}
\newcommand{\Enc}{\mathsf{Enc}}
\newcommand{\Dec}{\mathsf{Dec}}
\newcommand{\Sign}{\mathsf{Sign}}
\newcommand{\Ver}{\mathsf{Ver}}

\providecommand{\mypara}[1]{\smallskip\noindent\emph{#1} }
\providecommand{\myparab}[1]{\smallskip\noindent\textbf{#1} }
\providecommand{\myparasc}[1]{\smallskip\noindent\textsc{#1} }
\providecommand{\para}{\smallskip\noindent}


\newtheorem{theorem}{Theorem}
\theoremstyle{definition}
\newtheorem{ex}{Exercise}
\newtheorem{definition}{Definition}

%%%%%%%  Author Notes %%%%%%%d
%
\ifnum\shownotes=1
\newcommand{\authnote}[2]{{ $\ll$\textsf{\footnotesize #1 notes: #2}$\gg$}}
\else
\newcommand{\authnote}[2]{}
\fi
\newcommand{\Snote}[1]{{\authnote{Solution}{#1}}}
\newcommand{\Inote}[1]{{\authnote{Solution}{#1}}}
\newcommand{\Ichanged}[1]{{\authnote{Changed}{#1}}}
%%%%%%%%%%%%%%%%%%%%%%%%%%%%%%%%%

\newcommand{\VAR}{\mathrm{VAR}}



% end of macros
%%%%%%%%%%%%%%%%%%%%%%%%%%%%%%%%%%%%%%%%%%%%%%%%%%%%%%%%%%%%%%


% page counting, header/footer
\usepackage{fancyhdr}
\usepackage{lastpage}
\pagestyle{fancy}
\lhead{\footnotesize \parbox{11cm}{CS455, Boston University, Fall 2015} }
\rhead{Erik Brakke}
\renewcommand{\headheight}{24pt}

\begin{document}

\title{Homework 3}
\author{Erik Brakke}
\maketitle

\thispagestyle{fancy}


\section*{Answer 1}
\begin{enumerate}
	\item[(a)] $2MB = 2000KB$.  Segments double with each RTT.  $ceil(log_2(2000)) = 11$. It will take 12 RTT

	\item[(b)] $11 RTT$ to send $2^12KB = 4096KB = 4MB$.  We still have to send 12MB at 2MB per RTT, so $12 + (12/2) = 18RTT$

	\item[(c)] $18 RTT * 200ms = 3600ms$  This is a throughput of $16MB / 3.6s = 4.44MB/s$  Link uilizaion is 3.56\% (4.44MB/s * 8 / 1Gbps) 
\end{enumerate}

\section*{Answer 2}
Next hop table A\\
\begin{tabular}{|c|c|c|}
\hline
Dest & Dist & Next Hop\\
\hline
A & 0 & A\\
B & 7 & D\\
C & 6 & D\\
D & 3 & D\\
E & 5 & D\\
F & 12 & D\\
\hline
\end{tabular}

Next hop table B\\
\begin{tabular}{|c|c|c|}
\hline
Dest & Dist & Next Hop\\
\hline
A & 7 & E\\
B & 0 & B\\
C & 3 & E\\
D & 4 & E\\
E & 2 & E\\
F & 9 & E\\
\hline
\end{tabular}

Next hop table C\\
\begin{tabular}{|c|c|c|}
\hline
Dest & Dist & Next Hop\\
\hline
A & 6 & E\\
B & 3 & E\\
C & 0 & C\\
D & 3 & E\\
E & 1 & E\\
F & 6 & F\\
\hline
\end{tabular}

\section*{Answer 3}
\begin{tabular}{|c|c|c|}\hline
Dest & Dist & Next Hop\\
\hline
Net 1 & 0 & direct\\
Net 2 & 0 & direct\\
Net 5 & 8 & Router L\\
Net 17 & 6 & Router M\\
Net 22 & 9 & Router J\\
Net 24 & 6 & Router J\\
Net 30 & 2 & Router Q\\
Net 42 & 4 & Router J\\
\hline
\end{tabular}

\section*{Answer 4}
size of 1 cost vector = $8 bits * 60 = 480bits$\\
Each node sends $480 * 2 = 960bits$ per second\\
Each link sends $960 * 2 = 1920bits$ per second\\
If we assume that each link has $C$ capacity, then the capacity consumed is $1920/C$

\section*{Answer 5}
Yes fragmentation takes place because we have 2000 bytes, but can only send 262 bytes in a frame\\
$ceil(2000/242) = 9$ fragments would be sent with 242 bytes as the payload and 20 bytes as the header\\ 

\section*{Answer 6}
\begin{enumerate}
	\item[(a)]
	Network address: 145.98.0.0\\
	Subnet number: 145.98.128.0\\
	host number: 145.98.224.99\\

	\item[(b)] 145.98.128.0/17

	\item[(c)] There are $2^{11}$ addresses ranged 214.13.192.0 - 214.13.199.255

\end{enumerate}

\section*{Answer 7}
\begin{enumerate}
	\item[(a)] 
	\begin{tabular}{|c|c|}
	\hline
	Prefix & Next Hop\\
	\hline
	PA & C1.B3.0.0/16\\
	PB & C1.A0.0.0/12\\
	Q & C2.0.0.0/8\\
	\end{tabular}

	\item[(b)]
	\begin{tabular}{|c|c|}
	\hline
	Prefix & Next Hop\\
	\hline
	PA & C1.B3.0.0/16\\
	Q & C1.A0.0.0/12\\
	Q & C2.0.0.0/8\\
	R & C2.0B.10.0/20\\
	\end{tabular}
\end{enumerate}
\end{document} 
